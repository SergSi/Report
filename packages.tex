% Базовые настройки для русского языка
\usepackage{iftex}

\ifXeTeX
  \usepackage{fontspec}
  \setmainfont{Times New Roman}
\else
  \usepackage[T2A]{fontenc}
  \usepackage[utf8]{inputenc}
  \usepackage{cmap}
\fi

\usepackage[russian]{babel}
\usepackage{amsmath}
\usepackage{enumitem}
\usepackage{xcolor}
\usepackage{graphicx}
\usepackage{tabularx}
\usepackage{booktabs}
\usepackage{abstract}
\usepackage{geometry}
\usepackage{microtype}
\usepackage{advanced_image_template}
\usepackage{type1cm}
\usepackage{titlesec}
\usepackage[ampersand]{easylist}

% Полное отключение маркеров
\ListProperties(
    Hide=1000,               % Скрыть все уровни
    Progressive*=1.5em,      % Отступы без маркеров
    Style1*={},              % Пустой стиль для 1 уровня
    Style2*={},              % Пустой стиль для 2 уровня
    Style3*={}               % Пустой стиль для 3 уровня
)

\geometry{
  left=3cm,
  right=2cm,
  top=2cm,
  bottom=2cm,
  a4paper
}

% Настройки шрифта
\renewcommand{\normalsize}{\fontsize{13pt}{15.6pt}\selectfont}
\AtBeginDocument{
   \renewcommand{\small}{\fontsize{11pt}{13.2pt}\selectfont}
   \renewcommand{\footnotesize}{\fontsize{10pt}{12pt}\selectfont}
   \renewcommand{\large}{\fontsize{14pt}{16.8pt}\selectfont}
}

% Настройки списков
\setlist{
  topsep=0.5\baselineskip,
  partopsep=0pt,
  parsep=0pt,
  itemsep=0.25\baselineskip
}

\setlist[itemize]{%
  label={},
  leftmargin=1.5em,
  itemindent=0em,
  labelsep=0.5em,
  align=parleft
}

\setlist[enumerate]{%
  label={},
  leftmargin=2em,
  itemindent=0em,
  labelsep=0.5em
}

\setlist[description]{%
  labelindent=0em,
  leftmargin=2em,
  itemindent=0em,
  labelsep=0.5em,
  style=sameline,
  font=\normalfont
}

% Настройки переносов
\emergencystretch=3em
\hyphenpenalty=1000
\tolerance=500
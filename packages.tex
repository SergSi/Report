% ===== Базовые настройки для XeLaTeX =====
\usepackage{fontspec}
\usepackage{polyglossia}
\setmainlanguage{russian}
\setotherlanguage{english}

% ===== Основные шрифты =====
\setmainfont{Times New Roman}[
    UprightFont = *,
    BoldFont = * Bold,
    ItalicFont = * Italic,
    BoldItalicFont = * Bold Italic,
    Ligatures = TeX
]

% Определяем моноширинный шрифт с поддержкой кириллицы
\newfontfamily\cyrillicfonttt{Courier New} % Для Windows
% Альтернативы:
% \newfontfamily\cyrillicfonttt{DejaVu Sans Mono} % Для Linux
% \newfontfamily\cyrillicfonttt{PT Mono} % Если установлен

% ===== Математические пакеты =====
\usepackage{amsmath, amsfonts, amssymb, amsthm, mathtools}
\usepackage{physics}
\usepackage{esvect}
\usepackage{icomma}

% ===== Графика и таблицы =====
\usepackage{graphicx}
\usepackage{tabularx}
\usepackage{multirow}

% ===== Оформление документа =====
\usepackage[labelsep=period, justification=centering]{caption}
\captionsetup[table]{font=large}
\captionsetup[figure]{font=large}

% ===== Настройки страницы =====
\usepackage[
    left=2cm,
    right=2cm,
    top=2cm,
    bottom=2cm,
    headheight=1cm
]{geometry}

% ===== Колонтитулы =====
\usepackage{fancyhdr}
\fancyhf{}
\fancyhead[R]{\normalfont\bfseries \HeaderText}  
\fancyfoot[C]{\thepage}
\renewcommand{\headrulewidth}{0.4pt}
\pagestyle{fancy}

% ===== Компактные списки =====
\usepackage{enumitem}

% Основные настройки списков
\setlist{
    nosep,
    noitemsep,
    topsep=0pt,
    partopsep=0pt,
    labelsep=0.5em
}

% Специальные настройки для разных уровней вложенности
\setlist[enumerate,1]{
    leftmargin=35pt,    % Отступ для первого уровня (1., 2.)
    itemindent=0pt      % Отступ текста пунктов
}

\setlist[enumerate,2]{
    leftmargin=25pt,    % Увеличенный отступ для второго уровня (1.1, 1.2)
    itemindent=0pt
}

\setlist[enumerate,3]{
    leftmargin=25pt,    % Отступ для третьего уровня (а), б))
    itemindent=0pt
}

% ===== Настройки шрифта =====
\renewcommand{\baselinestretch}{1.25}
\renewcommand{\normalsize}{\fontsize{14}{16}\selectfont}


% ===== Гиперссылки =====
\usepackage[colorlinks, linkcolor=blue]{hyperref}

% ===== Прочие пакеты=====
\usepackage{indentfirst} % Обычно добавляет отступ первому абзацу
\makeatletter
\let\@afterindentfalse\@afterindentfalse
\def\@afterheading{%
  \@nobreaktrue
  \everypar{%
    \if@nobreak
      \@nobreakfalse
      \clubpenalty\@M
      \setbox\z@\lastbox % Удаляет отступ
    \else
      \clubpenalty\@clubpenalty
      \everypar{}%
    \fi}}
\makeatother

\usepackage{titling}
\usepackage{abstract}
\usepackage{tocloft}
\usepackage{setspace} % Для \setstretch

% Настройка \maketitle
\renewcommand{\maketitlehooka}{\setstretch{1.0}}

\usepackage{abstract}
\renewcommand{\abstracttextfont}{\normalfont\small\setstretch{1.0}}

% Настройка оглавления
\renewcommand{\cftbeforesecskip}{2pt}
\renewcommand{\cftbeforesubsecskip}{1pt}

% Настройка заголовков
\usepackage{titlesec}

\titleformat{\section}
    {\normalfont\large\bfseries}
    {\thesection.} % Точка для section
    {0.5em}
    {}

\titleformat{\subsection}
    {\normalfont\normalsize\bfseries}
    {\thesubsection.} % Точка для subsection
    {0.5em}
    {}

\titleformat{\subsubsection}
    {\normalfont\normalsize\itshape}
    {\thesubsubsection.} % Точка для subsubsection
    {0.5em}
    {}

% Отступы вокруг заголовков
\titlespacing*{\section}{0pt}{12pt}{6pt}
\titlespacing*{\subsection}{0pt}{8pt}{4pt}
\titlespacing*{\subsubsection}{0pt}{6pt}{2pt}
\begin{thebibliography}{9}
\bibitem{Butyrin2007}Бутырин А.Ю. \textit{Методики исследования объектов судебной строительно-технической экспертизы: межевых границ земельных участков. Методика утверждена научно-методическим советом Российского федерального центра судебной экспертизы при Минюсте России.}. М., 2007.    
\bibitem{GKINP-02-033-82}Инструкции по топографической съемке в масштабах 1:5000, 1:2000, 1:1000 и 1:500, ГКИНП-02-033-82.  
\bibitem{GKINP-02-262-02}Инструкции по развитию съемочного обоснования и съемке ситуации и рельефа с применением глобальных навигационных спутниковых систем ГЛОНАС и GPS ГКИНП (ОНТА)-02-262-02.  
\bibitem{PP985_2020}Постановление Правительства РФ от 04.07.2020 № 985 "Об утверждении перечня национальных стандартов и сводов правил (частей таких стандартов и сводов правил), в результате применения которых на обязательной основе обеспечивается соблюдение требований Федерального закона "Технический регламент о безопасности зданий и сооружений" и о признании утратившими силу некоторых актов Правительства Российской Федерации".  
\bibitem{Yalta2019}Правила землепользования и застройки территории муниципального образования городской округ Ялта РК, утвержденные решением внеочередной 85-й сессии 1-го созыва Ялтинского городского совета №16 от 19.07.2019.  
\bibitem{SNiP30-02-97}СНиП 30-02-97*. Планировка и застройка территорий садоводческих (дачных) объединений граждан, здания и сооружения (приняты и введены в действие Постановлением Госстроя РФ от 10.09.1997 № 18-51) (ред. от 12.03.2001).  
\bibitem{PP47_2006}Постановление Правительства РФ от 28.01.2006 № 47 (ред. от 27.07.2020) "Об утверждении Положения о признании помещения жилым помещением, жилого помещения непригодным для проживания, многоквартирного дома аварийным и подлежащим сносу или реконструкции, садового дома жилым домом и жилого дома садовым домом".  
\bibitem{SP14.13330.2018}СП 14.13330.2018. Свод правил. Строительство в сейсмических районах. Актуализированная редакция СНиП II-7-81* (утв. и введен в действие Приказом Минстроя России от 24.05.2018 № 309/пр) (ред. от 26.12.2019).  
\bibitem{SP17.13330.2017}СП 17.13330.2017. Свод правил. Кровли. Актуализированная редакция СНиП II-26-76 (утв. Приказом Минстроя России от 31.05.2017 № 827/пр) (ред. от 18.02.2019).  
\bibitem{SP19.13330.2019}СП 19.13330.2019. Свод правил. Сельскохозяйственные предприятия. Планировочная организация земельного участка (СНиП II-97-76* Генеральные планы сельскохозяйственных предприятий) (утв. и введен в действие Приказом Минстроя России от 14.10.2019 N 620/пр).  
\bibitem{SP30.13330.2016}СП 30.13330.2016. Свод правил. Внутренний водопровод и канализация зданий. СНиП 2.04.01-85* (утв. и введен в действие Приказом Минстроя России от 16.12.2016 № 951/пр) (ред. от 24.01.2019).  
\bibitem{SP30-102-99}СП 30-102-99. Планировка и застройка территорий малоэтажного жилищного строительства" (принят Постановлением Госстроя России от 30.12.1999 № 94).  
\bibitem{SP42.13330.2016}СП 42.13330.2016. Свод правил. Градостроительство. Планировка и застройка городских и сельских поселений. Актуализированная редакция СНиП 2.07.01-89* (утв. Приказом Минстроя России от 30.12.2016 № 1034/пр) (ред. от 19.12.2019).  
\bibitem{SP50.13330.2012}СП 50.13330.2012. Свод правил. Тепловая защита зданий. Актуализированная редакция СНиП 23-02-2003 (утв. Приказом Минрегиона России от 30.06.2012 № 265) (ред. от 14.12.2018).  
\bibitem{SP52.13330.2016}СП 52.13330.2016. Свод правил. Естественное и искусственное освещение. Актуализированная редакция СНиП 23-05-95*" (утв. Приказом Минстроя России от 07.11.2016 № 777/пр) (ред. от 20.11.2019).  
\bibitem{SP54.13330.2016}СП 54.13330.2016. Свод правил. Здания жилые многоквартирные. Актуализированная редакция СНиП 31-01-2003" (утв. Приказом Минстроя России от 03.12.2016 № 883/пр) (ред. от 19.12.2019).  
\bibitem{SP55.13330.2016}СП 55.13330.2016. Свод правил. Дома жилые одноквартирные. СНиП 31-02-2001 (утв. и введен в действие Приказом Минстроя России от 20.10.2016 № 725/пр).  
\bibitem{SP60.13330.2016}СП 60.13330.2016. Свод правил. Отопление, вентиляция и кондиционирование воздуха. Актуализированная редакция СНиП 41-01-2003 (утв. Приказом Минстроя России от 16.12.2016 № 968/пр) (ред. от 22.01.2019).  
\bibitem{SP70.13330.2012}СП 70.13330.2012. Свод правил. Несущие и ограждающие конструкции. Актуализированная редакция СНиП 3.03.01-87 (утв. Приказом Госстроя от 25.12.2012 № 109/ГС) (ред. от 26.12.2017).  
\bibitem{SP131.13330.2018}СП 131.13330.2018. Свод правил. Строительная климатология. СНиП 23-01-99 (утв. Приказом Минстроя России от 28.11.2018 № 763/пр).  
\bibitem{FZ123}Федеральный закон от 22.07.2008 № 123-ФЗ (ред. от 27.12.2018) "Технический регламент о требованиях пожарной безопасности".  
\bibitem{SP4.13130.2013}СП 4.13130.2013. Свод правил. Системы противопожарной защиты. Ограничение распространения пожара на объектах защиты. Требования к объемно-планировочным и конструктивным решениям (утв. Приказом МЧС России от 24.04.2013 № 288) (ред. от 14.02.2020).  
\bibitem{VSN53-86}ВСН 53-86(р). Правила оценки физического износа жилых зданий" (утв. Приказом Госгражданстроя от 24.12.1986 № 446).  
\bibitem{SanPiN2.1.3684-21}СанПиН 2.1.3684-21 Санитарно-эпидемиологические требования к содержанию территорий городских и сельских поселений, к водным объектам, питьевой воде и питьевому водоснабжению, атмосферному воздуху, почвам, жилым помещениям, эксплуатации производственных, общественных помещений, организации и проведению санитарно-противоэпидемических (профилактических) мероприятий.  
\bibitem{DBN360-92}Государственные строительные нормы Украины. ДБН 360 - 92. Градостроительство. Планировка и застройка городских и сельских поселений.  
\bibitem{DBNV.2.2-15-2005}Государственные строительные нормы Украины. Здания и сооружения. Жилые здания. Основные положения ДБН В.2.2-15-2005.  
\bibitem{DBNV.1.1-7-2002}Государственные строительные нормы Украины. В.1.1-7-2002. Пожарная безопасность объектов строительства.  
\bibitem{SNiP2.01.02-85}СНиП 2.01.02-85 Противопожарные нормы.  
\bibitem{SP113.13330.2016}СП 113.13330.2016. Свод правил. Стоянки автомобилей. Актуализированная редакция СНиП 21-02-99.  
\bibitem{SP13-102-2003}СП 13-102-2003. Правила обследования несущих строительных конструкций зданий и сооружений.  
\bibitem{GOSTR54851}ГОСТ Р 54851-2011. Конструкции строительные ограждающие неоднородные. Расчет приведенного сопротивления теплопередаче.  
\bibitem{Prikaz37_1998}Приказ Министерства Российской Федерации по земельной политике, строительству и жилищно-коммунальному хозяйству от 04.08.1998 года № 37.  
\end{thebibliography}
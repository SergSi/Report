\section*{Списки}
\begin{easylist}
& Перечень исследуемых документов (в хронологической последовательности):
&& -- правоустанавливающие документы;
&& -- правоудостоверяющие документы;
&& -- градостроительные документы;
&& -- документы, полученные из открытых источников;
&& -- другие документы, полученные от заказчика.
& Содержание документов
& Описание территории:
&& -- местоположение;
&& -- площадь;
&& -- категория земель;
&& -- вид разрешенного использования.
\end{easylist}

\begin{easylist}
& 1. Анализ текущего состояния земель: 
&& а) сбор данных, натурное обследование; 
&& б) изучение намерений;
&& в) проверка достоверности представленных материалов;
&& г) адаптация информации для заинтересованных лиц.
& 2. Выявление возможностей и ограничений (факторный анализ):
&& а) систематизация данных;
&& б) определение факторов, влияющих на землепользование; 
\end{easylist}

\begin{easylist}
& 1. Первый уровень (текст без маркера)
&& 1.1. Выделенные условия землепользования сравниваются между собой, выявляются несоответствия условий. Выявленные несоответствия оцениваются, систематизируются. Оценивается готовность
действующих лиц к преодолению ограничений.
&&& а) Третий уровень (детализация без маркера)
&&& б) Еще один пункт третьего уровня
&& 2.1 Другой подпункт второго уровня
& 2. Основной пункт без маркера
&& 2.1 Подраздел без маркера
\end{easylist}

\section*{Примеры изображений}

% Простой вариант
\insertimage{example-image.pdf}{Схема работы алгоритма}

% С дополнительными параметрами
\insertimage[width=0.5\textwidth,angle=90]{photo.jpg}{Пример повернутого изображения}

\paragraph{Параграф} Пример параграфа в документе. Текст параграфа может содержать различные элементы оформления.

\subparagraph{Подпараграф} Пример подпараграфа с дополнительными уточнениями.

\section*{Математические выражения}
\label{sec:math}

Пример математического выражения в тексте: $E = mc^2$.

Уравнение в выключной формуле:
\begin{equation}
    \label{eq:1}
    \int_{-\infty}^\infty e^{-x^2} dx = \sqrt{\pi}
\end{equation}

Система уравнений:
\[
\begin{cases}
    \frac{dx}{dt} = \sigma(y - x) \\
    \frac{dy}{dt} = x(\rho - z) - y \\
    \frac{dz}{dt} = xy - \beta z
\end{cases}
\]

\section*{Пример таблицы}
\label{sec:table}

\begin{table}[h]
    \centering
    \caption{Пример таблицы с границами}
    \label{tab:example}
    \begin{tabular}{|l|c|r|}
        \hline
        Заголовок 1 & Заголовок 2 & Заголовок 3 \\
        \hline
        Данные 1 & Данные 2 & Данные 3 \\
        \hline
        Данные 4 & Данные 5 & Данные 6 \\
        \hline
    \end{tabular}
\end{table}

В таблице \ref{tab:example} представлен пример оформления таблицы с границами.
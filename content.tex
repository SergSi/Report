% Заголовок, автор, дата
\title{
    \LARGE\textbf{Экспертное заключение} \\
    \large\textit{Возможности оформления пользования водными объектами на землях сельскохозяйственного назначения}
}

\author{
    \small\textbf{Иванов Иван Иванович} \\[4pt]
    \footnotesize Эксперт-землеустроитель \\
    \footnotesize Межрегиональная общественная организация \\
    \footnotesize «Национальный союз землепользователей» \\[8pt]
    \footnotesize {E-mail: info@nsz.su} \\
    \footnotesize {Тел.: +7 (915) 123-45-67}
}

\date{
    \small\textbf{10 апреля 2025~г.} \\[4pt]
}

\pagestyle{fancy}
\maketitle

\begin{abstract}
Экспертное исследование направлено на выявление возможностей оформления пользования водными объектами, расположенными на землях сельскохозяйственного назначения. Особое внимание уделено учету сельскохозяйственных угодий в информационной системе Министерства сельского хозяйства Российской Федерации. 
\end{abstract}

\tableofcontents

\section{Введение}
\label{sec:intro}

\begin{enumerate}
    \item[] Цель экспертного исследования
    \item[] Задачи экспертного исследования
    \item[] Задачи экспертного исследования, определенные экспертом 
    \item[] Нормативно-правовая база исследования
\end{enumerate}

\section{Объект исследования}
\label{sec:obj}
Объектом исследования служат документы и территория. 

\begin{enumerate}
    \item[1.] Перечень исследуемых документов (в хронологической последовательности):
    \begin{enumerate}
        \item[--] правоустанавливающие документы;
        \item[--] правоудостоверяющие документы;
        \item[--] градостроительные документы;
        \item[--] документы, полученные из открытых источников;
        \item[--] другие документы, полученные от заказчика.
    \end{enumerate}
    \item[2.] Содержание документов
    \item[3.] Описание территории:
    \begin{enumerate}
        \item[--] местоположение;
        \item[--] площадь;
        \item[--] категория земель;
        \item[--] вид разрешенного использования.
    \end{enumerate}
\end{enumerate}

\section{Экспертная оценка}
\label{sec:expert}

\subsection{Методы проведения исследования}
\label{subsec:method}

Для достижения целей исследования и решения поставленных задач использованы методы:

\begin{enumerate}
    \item[1.] Анализ текущего состояния земель: 
    \begin{enumerate}
        \item[а)] сбор данных, натурное обследование; 
        \item[б)] изучение намерений;
        \item[в)] проверка достоверности представленных материалов;
        \item[г)] адаптация информации для заинтересованных лиц.
    \end{enumerate}
    \item[2.] Выявление возможностей и ограничений (факторный анализ):
    \begin{enumerate}
        \item[а)] систематизация данных;
        \item[б)] определение факторов, влияющих на землепользование; 
        \item[в)] выявление расхождений между представленными материалами и реальностью;
        \item[г)] образовательные мероприятия; 
        \item[д)] оценка готовности действующих лиц.
    \end{enumerate}
    \item[3.] Подготовка предложений (при подтвержденной готовности): 
    \begin{enumerate}
        \item[а)] разработка мероприятий, 
        \item[б)] консультации с заинтересованными лицами, корректировка целей и намерений, формирование рационального подхода, оформление отчета.
    \end{enumerate}
\end{enumerate}

\subsection{Анализ условий землепользования}
\label{subsec:zem}

\begin{enumerate}
    \item[1.] Природно-ресурсные условия
    \begin{enumerate}
        \item[1.1.] Природно-географические условия:
        \begin{enumerate}
            \item[а)] рельеф;
            \item[б)] климат;
            \item[в)] почвы;
            \item[г)] водные ресурсы.
        \end{enumerate}
        \item[1.2.] Мелиоративное состояние земель:
        \begin{enumerate}
            \item[а)] наличие мелиоративных систем;
            \item[б)] состояние мелиоративных систем.
        \end{enumerate}
    \end{enumerate}
    \item[2.] Экономические условия:
    \begin{enumerate}
        \item[а)] доходность;
        \item[б)] инвестиционная привлекательность;
        \item[в)] наличие финансовых ресурсов.
    \end{enumerate}
    \item[3.] Социальные условия:
    \begin{enumerate}
        \item[а)] наличие трудовых ресурсов;
        \item[б)] уровень жизни населения;
        \item[в)] общественное мнение по поводу развития территории.
    \end{enumerate}
    \item[4.] Территориально-планировочные условия
    \begin{enumerate}
        \item[4.1.] Градостроительное зонирование:
        \begin{enumerate}
            \item[а)] наличие зон с особыми условиями использования;
            \item[б)] территориальные зоны;
            \item[в)] элементы планировочной структуры.
        \end{enumerate}
        \item[4.2.] Образование земельных участков:
        \begin{enumerate}
            \item[а)] наличие определенных границ земельных участков;
            \item[б)] обеспечение доступа к землям общего пользования, подъездных путей;
            \item[в)] образование земельных участков общего назначения;
            \item[г)] соответствие фактического использования земель их целевому назначению и разрешенному использованию.
        \end{enumerate}
    \end{enumerate}
    \item[5.] Правовые условия
    \begin{enumerate}
        \item[5.1.] Возникшие права и обязанности:
        \begin{enumerate}
            \item[а)] права собственности;
            \item[б)] аренда;
            \item[в)] сервитуты.
        \end{enumerate}
        \item[5.2.] Установленные ограничения прав:
        \begin{enumerate}
            \item[а)] ограничения на использование земель в границах зон с особыми условиями использования.
        \end{enumerate}
    \end{enumerate}
    \item[6.] Инфраструктурные условия
    \begin{enumerate}
        \item[6.1.] Инженерное оборудование территории:
        \begin{enumerate}
            \item[а)] наличие и состояние инженерных сетей (электричество, водоснабжение, канализация).
        \end{enumerate}
        \item[6.2.] Благоустройство территории:
        \begin{enumerate}
            \item[а)] наличие зеленых зон, парков, мест отдыха;
            \item[б)] гостевые автостоянки.
        \end{enumerate}
    \end{enumerate}
\end{enumerate}

Выделенные условия землепользования описывают, сравнивают между собой, выявляются несоответствия условий. Выявленные несоответствия оцениваются, систематизируются.

\subsection{Оценка действующих лиц}
\label{subsec:lica}

\begin{enumerate}
    \item[1.] Состав действующих лиц:
    \begin{enumerate}
        \item[а)] землепользователи;
        \item[б)] инвесторы;
        \item[в)] застройщики;
        \item[г)] органы власти;
        \item[д)] местные жители.
    \end{enumerate}
    \item[2.] Возможности действующих лиц:
    \begin{enumerate}
        \item[а)] финансовые ресурсы;
        \item[б)] технические ресурсы;
        \item[в)] человеческие ресурсы.
    \end{enumerate}
    \item[3.] Интересы и намерения действующих лиц:
    \begin{enumerate}
        \item[а)] намерения по использованию земель;
        \item[б)] соответствие намерений реальным возможностям;
        \item[в)] реализованные и нереализованные интересы.
    \end{enumerate}
\end{enumerate}

Оценивается готовность действующих лиц к преодолению ограничений.

\section{Выводы}
\label{sec:end}

После проведения анализа эксперт формулирует выводы о выявленных условиях, потенциале территории и проблемах.

\subsection{Проблемы и ограничения}
\label{subsec:problem}

\begin{enumerate}
    \item[1.] Критические условия (препятствия, требующие обязательных действий):
    \begin{enumerate}
        \item[а)] отсутствие подъездных путей;
        \item[б)] экологические проблемы: загрязнение почв, загрязнение водных объектов;
        \item[в)] отсутствие инженерных коммуникаций.
    \end{enumerate}
    \item[2.] Регулируемые условия (факторы, поддающиеся корректировке):
    \begin{enumerate}
        \item[а)] изменение вида разрешенного использования;
        \item[б)] развитие инфраструктуры территории;
        \item[в)] оптимизация границ землепользования.
    \end{enumerate}
    \item[3.] Несоответствие целевому назначению:
    \begin{enumerate}
        \item[а)] участки, используемые не по назначению.
    \end{enumerate}
    \item[4.] Экологические ограничения:
    \begin{enumerate}
        \item[а)] загрязнение почвенного покрова;
        \item[б)] деградация водных ресурсов;
        \item[в)] нарушение природного ландшафта.
    \end{enumerate}
    \item[5.] Экономические ограничения:
    \begin{enumerate}
        \item[а)] низкая рентабельность землепользования;
        \item[б)] недостаток инвестиционных вложений;
        \item[в)] высокая стоимость подключения к инженерным сетям.
    \end{enumerate}
    \item[6.] Инфраструктурные ограничения:
    \begin{enumerate}
        \item[а)] отсутствие уличной и дорожной сети;
        \item[б)] нехватка инженерных коммуникаций;
        \item[в)] отсутствие доступа к землям общего пользования.
    \end{enumerate}
    \item[7.] Правовые ограничения:
    \begin{enumerate}
        \item[а)] наличие зон с особыми условиями использования земель;
        \item[б)] ошибки в кадастровых записях;
        \item[в)] несоответствие документации фактическому использованию.
    \end{enumerate}
    \item[8.] Организационные проблемы:
    \begin{enumerate}
        \item[а)] отсутствие координации между участниками процесса;
        \item[б)] недостаточный уровень взаимодействия с органами власти;
        \item[в)] неэффективное градостроительное планирование;
        \item[г)] отсутствие землеустройства.
    \end{enumerate}
\end{enumerate}

\textbf{Выявленные проблемы и несоответствия}

\subsection{Потенциал территории}
\label{subsec:potenc}

\begin{enumerate}
    \item[1.] Возможности для развития:
    \begin{enumerate}
        \item[а)] изменение функционального назначения отдельных участков;
        \item[б)] повышение инвестиционной привлекательности; 
        \item[в)] увеличение отдачи от вложенных ресурсов.
    \end{enumerate}
    \item[2.] Наличие ресурсов для реализации планов:
    \begin{enumerate}
        \item[а)] финансовые возможности действующих лиц;
        \item[б)] техническая оснащенность;
        \item[в)] кадровый потенциал.
    \end{enumerate}
\end{enumerate}

\section{Рекомендации}
\label{sec:rek}

\begin{enumerate}
    \item[1.] По изменению видов использования:
    \begin{enumerate}
        \item[а)] предлагаемые варианты;
        \item[б)] обоснование изменений;
    \end{enumerate}
    \item[2.] По инфраструктурным улучшениям:
    \begin{enumerate}
        \item[а)] дорожная сеть;
        \item[б)] инженерные коммуникации;
    \end{enumerate}
    \item[3.] По охране окружающей среды;
    \item[4.] По правовому сопровождению;
    \item[5.] По взаимодействию заинтересованных лиц.
\end{enumerate}

\section*{Приложения}
\label{sec:pril}
\addcontentsline{toc}{section}{Приложения}

\begin{enumerate}
    \item[1.] Картографические материалы:
    \begin{enumerate}
        \item[а)] схема границ территории;
        \item[б)] зонирование;
        \item[в)] инфраструктура;
    \end{enumerate}
    \item[2.] Фотоматериалы.
    \item[3.] Копии правоустанавливающих документов.
    \item[4.] Таблицы аналитических данных.
    \item[5.] Графики динамики изменений.
\end{enumerate}


% Заголовок, автор, дата
\title{
    \LARGE\textbf{Экспертное заключение} \\
    \large\textit{Возможности оформления пользования водными объектами на землях сельскохозяйственного назначения}
}

\author{
    \small\textbf{Иванов Иван Иванович} \\[4pt]
    \footnotesize Эксперт-землеустроитель \\
    \footnotesize Межрегиональная общественная организация \\
    \footnotesize «Национальный союз землепользователей» \\[8pt]
    \footnotesize {E-mail: info@nsz.su} \\
    \footnotesize {Тел.: +7 (915) 123-45-67}
}

\date{
    \small\textbf{10 апреля 2025~г.} \\[4pt]
}

\maketitle

%  Аннотация
\begin{abstract}
Экспертное исследование направлено на выявление возможностей оформления пользования водными объектами, расположенными на землях сельскохозяйственного назначения. Особое внимание уделено учету сельскохозяйственных угодий в информационной системе Министерства сельского хозяйства Российской Федерации. 
\end{abstract}

% Содержание
\tableofcontents


\section{Введение}
\label{sec:intro}

\begin{easylist}
& Цель экспертного исследования. 
& Задачи экспертного исследования, поставленные заказчиком.
& Задачи экспертного исследования, определенные экспертом. 
& Нормативно-правовая база исследования.
\end{easylist}

\section{Объект исследования}
\label{sec:obj}
\label{sec:object}
Объектом исследования служат документы и территория. 
\begin{easylist}
& Перечень исследуемых документов (в хронологической последовательности):
&& -- правоустанавливающие документы;
&& -- правоудостоверяющие документы;
&& -- градостроительные документы;
&& -- документы, полученные из открытых источников;
&& -- другие документы, полученные от заказчика.
& Содержание документов
& Описание территории:
&& -- местоположение;
&& -- площадь;
&& -- категория земель;
&& -- вид разрешенного использования.
\end{easylist}

\section{Экспертная оценка}
\label{sec:expert}
\label{sec:[expert]}

\subsection{Методы проведения исследования}
\label{subsec:[method]}

Для достижения целей исследования и решения поставленных задач использованы методы:
\begin{easylist}
& 1. Анализ текущего состояния земель: 
&& а) сбор данных, натурное обследование; 
&& б) изучение намерений;
&& в) проверка достоверности представленных материалов;
&& г) адаптация информации для заинтересованных лиц.
& 2. Выявление возможностей и ограничений (факторный анализ):
&& а) систематизация данных;
&& б) определение факторов, влияющих на землепользование; 
&& в) выявление расхождений между представленными материалами и реальностью;
&& г) образовательные мероприятия; 
&& д) оценка готовности действующих лиц.
& 3. Подготовка предложений (при подтвержденной готовности): 
&& а) разработка мероприятий, 
&& б) консультации с заинтересованными лицами, корректировка целей и намерений, формирование рационального подхода, оформление отчета.
\end{easylist}

\subsection{Анализ условий землепользования}
\label{subsec:zem}
\begin{easylist}
& 1. Природно-ресурсные условия
&& 1.1. Природно-географические условия:
&&& а) рельеф;
&&& б) климат;
&&& в) почвы;
&&& г) водные ресурсы.
&& 1.2. Мелиоративное состояние земель:
&&& а) наличие мелиоративных систем;
&&& б) состояние мелиоративных систем.
& 2. Экономические условия:
&&& а) доходность;
&&& б) инвестиционная привлекательность;
&&& в) наличие финансовых ресурсов.
& 3. Социальные условия:
&&& а) наличие трудовых ресурсов;
&&& б) уровень жизни населения;
&&& в) общественное мнение по поводу развития территории.
& 4. Территориально-планировочные условия
&& 4.1. Градостроительное зонирование:
&&& а) наличие зон с особыми условиями использования;
&&& б) территориальные зоны;
&&& в) элементы планировочной структуры.
&& 4.2. Образование земельных участков:
&&& а) наличие определенных границ земельных участков;
&&& б) обеспечение доступа к землям общего пользования, подъездных путей;
&&& в) образование земельных участков общего назначения;
&&& г) соответствие фактического использования земель их целевому назначению и разрешенному использованию.
& 5. Правовые условия
&& 5.1. Возникшие права и обязанности:
&&& а) права собственности;
&&& б) аренда;
&&& в) сервитуты.
&& 5.2. Установленные ограничения прав:
&&& а) ограничения на использование земель в границах зон с особыми условиями использования.
& 6. Инфраструктурные условия
&& 6.1. Инженерное оборудование территории:
&&& а) наличие и состояние инженерных сетей (электричество, водоснабжение, канализация).
&& 6.2. Благоустройство территории:
&&& а) наличие зеленых зон, парков, мест отдыха;
&&& б) гостевые автостоянки.
\end{easylist}
Выделенные условия землепользования описывают, сравнивают между собой, выявляются несоответствия условий. Выявленные несоответствия оцениваются, систематизируются. 

\subsection{Оценка действующих лиц}
\label{subsec:lica}
\begin{easylist}
& 1. Состав действующих лиц:
&& а) землепользователи;
&& б) инвесторы;
&& в) застройщики;
&& г) органы власти;
&& д) местные жители.
& 2. Возможности действующих лиц:
&& а) финансовые ресурсы;
&& б) технические ресурсы;
&& в) человеческие ресурсы.
& 3. Интересы и намерения действующих лиц:
&& а) намерения по использованию земель;
&& б) соответствие намерений реальным возможностям;
&& в) реализованные и нереализованные интересы.
\end{easylist}
Оценивается готовность действующих лиц к преодолению ограничений.

\section{Выводы}
\label{sec:end}
После проведения анализа эксперт формулирует выводы о выявленных условиях, потенциале территории и проблемах.
\subsection{Проблемы и ограничения}
\label{subsec:problem}
\begin{easylist}
& 1. Критические условия (препятствия, требующие обязательных действий):
&& а) отсутствие подъездных путей;
&& б) экологические проблемы: загрязнение почв, загрязнение водных объектов;
&& в) отсутствие инженерных коммуникаций.
& 2. Регулируемые условия (факторы, поддающиеся корректировке):
&& а) изменение вида разрешенного использования;
&& б) развитие инфраструктуры территории;
&& в) оптимизация границ землепользования.
& 3. Несоответствие целевому назначению:
&& а) участки, используемые не по назначению.
& 4. Экологические ограничения:
&& а) загрязнение почвенного покрова;
&& б) деградация водных ресурсов;
&& в) нарушение природного ландшафта.
& 5. Экономические ограничения:
&& а) низкая рентабельность землепользования;
&& б) недостаток инвестиционных вложений;
&& в) высокая стоимость подключения к инженерным сетям.
& 6. Инфраструктурные ограничения:
&& а) отсутствие уличной и дорожной сети;
&& б) нехватка инженерных коммуникаций;
&& в) отсутствие доступа к землям общего пользования.
& 7. Правовые ограничения:
&& а) наличие зон с особыми условиями использования земель;
&& б) ошибки в кадастровых записях;
&& в) несоответствие документации фактическому использованию.
& 8. Организационные проблемы:
&& а) отсутствие координации между участниками процесса;
&& б) недостаточный уровень взаимодействия с органами власти;
&& в) неэффективное градостроительное планирование;
&& г) отсутствие землеустройства.
\end{easylist}

\textbf{Выявленные проблемы и несоответствия}

\begin{easylist}

\end{easylist}
\subsection{Потенциал территории}
\label{subsec:potenc}
\begin{easylist}
& 1. Возможности для развития:
&& а) изменение функционального назначения отдельных участков;
&& б) повышение инвестиционной привлекательности; 
&& в) увеличение отдачи от вложенных ресурсов.
& 2. Наличие ресурсов для реализации планов:
&& а) финансовые возможности действующих лиц;
&& б) техническая оснащенность;
&& в) кадровый потенциал.
\end{easylist}
\section{Рекомендации}
\label{sec:rek}
\begin{easylist}
& 1. По изменению видов использования:
&& а) предлагаемые варианты;
&& б) обоснование изменений;
& 2. по инфраструктурным улучшениям:
&& а) дорожная сеть;
&& б) инженерные коммуникации;
& 3. По охране окружающей среды;
& 4. По правовому сопровождению;
& 5. По взаимодействию заинтересованных лиц.
\end{easylist}
\section*{Приложения}
\label{sec:pril}
\addcontentsline{toc}{section}{Приложения}

\begin{easylist}
& 1. Картографические материалы:
&& а) схема границ территории;
&& б) зонирование;
&& в) инфраструктура;
& 2. Фотоматериалы.
& 3. Копии правоустанавливающих документов.
& 4. Таблицы аналитических данных.
& 5. Гафики динамики изменений.
\end{easylist}